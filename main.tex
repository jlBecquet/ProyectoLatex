\documentclass{article}
\usepackage[utf8]{inputenc}
\usepackage[spanish]{babel}
\usepackage{listings}
\usepackage{graphicx}
\graphicspath{ {images/} }
\usepackage{cite}

\begin{document}

\begin{titlepage}
    \begin{center}
        \vspace*{1cm}
            
        \Huge
        \textbf{Manejo de LaTeX}
            
        \vspace{0.5cm}
        \LARGE
        Instrucciones secuenciales
            
        \vspace{2.5cm}
            
        \textbf{Juan Luis Becquet Martínez}
            
        \vfill
            
        \vspace{0.8cm}
            
        \Large
        Ingeniería de Telecomunicaciones\\
        Universidad de Antioquia\\
        Sede Turbo\\
        Marzo de 2021
            
    \end{center}
\end{titlepage}

\tableofcontents
\newpage
\section{Introducción.}\label{intro}
En este proceso guiamos de manera procedimental y secuencial dando instrucciones claras, cortas y concisas para llegar finalmente al resultado esperado de esta actividad.

\section{Pasos detallados a realizar} \label{contenido}
        \textbf{Materiales}
        \vspace{0.4cm}
        
1. Conseguir 2 tarjetas plasticas, de preferencia 2 cedulas, tienen el tamaño y forma perfecta.
2.Conseguir una hoja de papel.

        \vspace{0.4cm}
        \textbf{Instrucciones}


Nota: De aqui en adelante las instrucciones se deberán realizar con una sola mano, de preferencia, su mano mas diestra, además las 2 tarjetas deben estar debajo de la hoja de papel y a la vez la hoja y las 2 tarjetas en una superficie plana al iniciar las instrucciones.\\
1.  Retire la hoja de papel de encima de las tarjetas y pongala a un lado en su superficie plana.\\
2.  con una unica mano alinie las tarjetas, todas sus esquinas deben coincidir(una encima de la otra).\\
3.  Agarre ambas tarjetas y coloquelas perpendicularmente(de manera vertical) en la superficie plana sin separarlas y sin soltarlas.\\
4.  apoye su dedo indice en la parte superior de ambas tarjetas en forma perdendicular, ubiquelo en el centro de las tarjetas y no lo retire hasta que se mencione.\\
5. con sus dedos pulgar, anular y corazon tome la tarjeta mas cercana a la parte interna de la palma de su mano y deslice la parte inferior hasta una distancia aproximada a la mitad del alto de las tarjetas para realizar una figura parecida a un triangulo.\\
6. retire los dedos suavemente de las tarjetas que deben quedar en completo equilibrio.\\
7.  si todo lo anterior falla, repita desde el paso 2 hasta el paso 5 pero graduando la distancia mencionada en el paso 5 a conveniencia.

\section{Conclusiones} \label{contenido}

        \vspace{0.4cm}

Para la conclusión positiva de tal actividad podemos concluir que las ordenes deben darse de una manera clara, restringiendo posibilidades al agente que realiza las acciones, sino estas van a terminar muy mal puesto que hay cabida a muchos errores e interpretaciones, además se debe de usar lenguaje claro ya que si se usa lenguaje muy tecnico es posible que las instrucciones no se entiendan.





\bibliographystyle{IEEEtran}
\bibliography{references}

\end{document}
